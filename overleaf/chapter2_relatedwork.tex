\chapter{Related Work}

In this chapter, we first discuss how the concept of \textbf{multimodality}, which describes how instructors can use a combination of different approaches and technologies to offer class content to students, has developed since the internet first started to become mainstream in the late 1990's. From there, we examine the works that have investigated the differences between in-person and remote learning, as well as synchronous and asynchronous learning. Then, we discuss how the shift to online learning during the COVID-19 pandemic has affected students' perceptions of hybrid learning and why instructors should be cautious when developing new hybrid courses. Finally, we examine what prior research has found on the impact of the HyFlex model and the flipped classroom model when they are used in college courses, as well as how the intersection between the two has yet to be explored in much depth.

\section{Multimodality Over the Years}

Researchers were conducting studies on the impact of multimodality in college courses long before the onset of COVID-19 in early 2020. In 1999, Latchman et al. \cite{latchman1999information} observed the efficacy of a course that was strikingly similar in design to much more recent online courses, even with the limits of the technology at that time. Unable to rely on a video conferencing tool like the online courses of today, it used an in-house method to digitize the video and audio of lectures for a local server to stream through a web browser, which also included a chat room through which students could communicate with the instructor; it also allowed students to learn the material asynchronously through video recordings of the lectures. The study found that students who attended ``synchronously in time but asynchronously in space both on and off campus" benefited from the ability to view lectures as many times as necessary through the recordings, while students who attended asynchronously were able to enroll in a course that they could not have engaged with had an asynchronous modality not been offered. This course can be seen as a precursor to contemporary remote courses that are facilitated by the use of tools like Zoom, and it shows that even limited remote and asynchronous modalities can have positive effects on student learning.

As technology progressed and access to the internet became more widespread, courses that combined multiple modalities into a single offering became a hot topic in education research. These hybrid courses, which combined elements of in-person, remote, and asynchronous learning, were often compared with courses that only utilized one of the three. El Mansour and Mupinga \cite{el2007students} interviewed 12 students enrolled in a hybrid industrial technology course and 41 students enrolled in an online industrial technology course to learn about their positive and negative experiences with their respective modalities. A larger percentage of students had negative experiences with the online course, and the online learners expressed that they felt that they had not been able to get to know their instructor well enough throughout the course. A decade later, Liu et al. \cite{liu2016effectiveness} conducted a meta-analysis of 56 studies that observed how effective hybrid courses had been at training individuals for health professions. They found that hybrid learning consistently had a positive effect on medical students, who experienced improved knowledge acquisition compared to their peers in both traditional in-person courses and online courses.

However, not all findings on hybrid learning were positive. Jackson and Helms \cite{jackson2008student} found that students in a business course reported an almost equal number of strengths and weaknesses with the hybrid model, with flexibility being cited as both a strength and a weakness at the same time; students liked being able to choose how to engage with the course, but they disliked the fact that choosing not to attend in person prevented them from interacting one-on-one with the course faculty. In the field of computing education research, Basu et al. \cite{basu2021online} conducted a study just before the COVID-19 pandemic occurred on a web-development course that was redesigned from its existing in-person version to a new hybrid offering. They found no significant difference in student engagement between the hybrid and in-person sections, since online students were more active on the course's discussion platform and in-person students submitted a higher number of assignments on time. It is unclear why different studies yield such different results when comparing hybrid courses to other courses, but this does show that a hybrid course is neither more nor less effective than its non-hybrid equivalent. It is worth investigating why the combination of several different modalities into a single course can result in complications that are not seen in courses that only implement one of those modalities.

\section{Differences Between Modalities}

One way that researchers have delved deeper into the nature of hybrid courses has been to compare the strengths and weaknesses of individual modalities. Numerous studies have been conducted on the differences between in-person and online learning, for instance. Marchand and Gutierrez \cite{marchand2012role} measured the effects of graduate students’ emotions on their learning process in an introductory research methods course that was taken in either an in-person or an online learning environment. Positive emotions such as optimism were found to be strong predictors of effective learning in all students, but to their surprise, both frustration and anxiety could be used to predict the effectiveness of in-person students' learning, while neither could be used to predict that of online students. In-person students with higher levels of frustration were more likely to use less effective learning strategies, while higher levels of course-related anxiety predicted more effective strategy use; given that the average level of anxiety among the face-to-face learners was low, they reasoned that the ``higher" levels of anxiety actually reflected more moderate anxiety levels, and were thus more likely to enhance the strategies that those students used in their learning. Meanwhile, they hypothesized that the ability for online students to regulate their own learning enabled them to diffuse negative emotions through more frequent breaks and time to reflect, reducing the effect of these emotions for better and for worse. Bali and Liu \cite{bali2018students} found that students in hybrid courses in the field of humanities and social sciences perceived in-person learning to be more satisfying than online learning, but they found no other meaningful differences in students’ perceptions of the two modalities. Many of the students in their study reported that they chose online learning over in-person learning because the sheer convenience and flexibility of the former outweighed their greater sense of satisfaction in the latter.

Studies have also drawn comparisons between synchronous and asynchronous modalities. Offir et al. \cite{offir2008surface} found synchronous online learning to be more effective for students with high cognitive ability than for those with low cognitive ability, which they reasoned was due to the limited nature of the interactions between instructors and students who do not ask questions in remote lectures. They also found that students with high cognitive ability were more easily able to overcome the distance between their own level of understanding and that of the instructor in asynchronous learning — despite its lack of interactivity — indicating that asynchronous students must actively push themselves to engage with a course in order to successfully learn. In turn, Sharifrazi and Stone \cite{sharifrazi2019students} found business students’ perceptions of synchronous online learning to be more positive than their perceptions of asynchronous learning, with some students citing positive moments when they used live chats to engage in dialogue with their instructors and receive personal feedback to their questions. They also note that even asynchronous online environments require an instructor to maintain an active presence and build the dialogue between students of the course. This further supports the idea that asynchronous learners also benefit from some level of social interaction, especially in fields like business that place an emphasis on the development of social skills.

\section{Current Perceptions of Hybrid Learning}

The COVID-19 pandemic had an undeniable effect on the state of research on these types of courses: as more higher learning institutions have adopted alternative modalities of instruction, a number of studies that investigate the effects of these modalities on their students have followed. Among these studies, there have been positive reports that students' increased exposure to hybrid courses has resulted in them being much more receptive to the hybrid model. Mali and Lim \cite{mali2021students} used a mixed-method approach to gather data on the perceptions that accounting students had towards hybrid and in-person learning a year after the start of the pandemic. Though students reported that they enjoyed in-person learning more than hybrid learning in cases when COVID-19 was not an issue, they also considered hybrid learning to be an enjoyable learning experience in these cases. Finlay et al. \cite{finlay2022virtual} observed a sport and exercise science program that used both hybrid and online learning during the pandemic, and they found a clear preference for hybrid learning among students. Students' satisfaction levels were also much higher in the hybrid learning environment, with hybrid students viewing the teaching, support, management, and community of the course much more positively than their online peers.

In some cases, instructors have been found to support hybrid learning at least as much as students have. Atwa et al. \cite{atwa2022online} conducted a mixed-method study on medical students and faculty members to gather their perceptions of hybrid learning. While 53.1\% of the medical students preferred in-person learning, 60.6\% of the faculty members preferred hybrid learning because it merges the self-paced learning of online modalities with the deeper and more practical discussion of in-person modalities. Lapitan et al. \cite{lapitan2021effective} directly observed the effectiveness of a hybrid offering of an undergraduate chemistry course that was developed during the pandemic. Not only were most students satisfied with the new structure of the course, instructors began to offer more remote and asynchronous options after being exposed to new technological teaching tools and even collaborated among themselves to develop these resources, showing that instructors have also directly benefited from being exposed to different course models. The sudden mass exposure of students and instructors to hybrid learning has led to more individuals understanding the unique benefits of the model, and as the study by Lapitan et al. indicates, this is likely to result in an increase in the number of hybrid courses that will be offered at higher learning institutions.

\section{Conditions for Effective Hybrid Courses}

However, researchers are still learning what conditions are necessary for a hybrid course to be effective. Bamoallem and Altarteer \cite{bamoallem2022remote} investigated the impact of three elements of instruction: teaching presence, which was defined as the design and facilitation of material by the instructor; cognitive presence, which was defined as the ability to develop one's knowledge through activities and discussion; and social presence, which was defined as the ability to establish relationships within the larger community of the course. They found that all three elements were strong indicators that students would respond positively to a hybrid course. Alammary \cite{alammary2019blended} conducted a review of several studies that had applied hybrid learning to introductory computer science courses. Though they found that the vast majority of studies showed that hybrid learning has a positive effect on teaching, they also identified five distinct subcategories of the hybrid learning model that had been utilized by these studies: a flipped model, a mixed model, a flex model, a supplemental model, and an online-practicing model. Studies like these show how simply adding modalities to an existing course will not necessarily lead to a high-quality educational experience for students. If hybrid courses are on the rise, instructors should be able to identify which aspects of these courses result in a more positive learning experience for students, and they must decide which applications of hybrid learning would best suit the needs of their students.

\section{HyFlex Courses}

One such application of hybrid learning is the ``hybrid-flexible" (HyFlex) model, which has multiple modalities through which students can attend lectures while not requiring attendance in any specific modality: students can choose to attend in-person in a traditional classroom setting, remotely through an online video conferencing tool, or asynchronously through a self-paced \textbf{learning management system (LMS)}. An LMS is an online platform that mainly serves as a source of learning materials, but instructors can also use them to distribute assignments and to facilitate online discussions between students. In the context of HyFlex courses, instructors often upload video recordings of lectures to an LMS so that students can attend them asynchronously and rewatch them later. Beatty \cite{beatty2014hybrid}, the originator of the model, encouraged instructors of HyFlex courses to prioritize four attributes: Learner Choice, Equivalency, Reusability, and Accessibility. Learner Choice provides students with the flexibility to choose whichever modality of attendance works best for them; Equivalency ensures that students of different modalities are participating in an equal number of activities; Reusability allows students of all modalities to access shared resources that support their learning; and Accessibility makes it so that students are able to receive an equitable level of access and support in the modality of their choosing. HyFlex courses that keep all of these aspects intact are more likely to provide a positive learning experience. However, HyFlex as a model is not without its downsides. Han et al. \cite{han2022students} investigated students' responses to a HyFlex web programming course, finding that they far preferred HyFlex over online learning. Despite this, one of their main challenges with HyFlex learning was that they had found it hard to breach the distance between themselves and students of different modalities, and the instructor had sometimes inadvertently prioritized one modality at the expense of the other during lectures. Unfortunately, though instructors can attempt to make the experience across different modalities as equitable as they can, desynchronization appears to be an ever-present issue in HyFlex courses.

\section{Flipped Classrooms}

The flipped model that Alammary defined in their review of studies on hybrid CS courses refers to the ``flipped classroom" model, which gained traction in the 2010's as a type of course in which students and instructors each play distinct roles in the learning process: students are tasked with watching lecture-style videos on course concepts before attending the in-person lectures, during which instructors lead a more in-depth discussion of the concepts and answer questions that students came up with while watching the videos. Campbell et al. \cite{campbell2014evaluating} found that their flipped offering of an introductory computer science course yielded high completion rates for the lecture preparation material but low attendance rates for the actual lectures, likely due to the fact that credit was given for preparatory work but not for lecture attendance. They also found that fewer students perceived lectures to be helpful than students in the traditional version of the course that was offered the semester prior, though the flipped course students also experienced increased enthusiasm and enjoyment compared to other courses they had taken. These findings indicate that students are less likely to engage synchronously with a course if they are given the ability to access in-depth resources at any time. This is further compounded by the conclusions made by K\"{o}ppe et al. \cite{koppe2015flipped} from five years of flipped courses: students who have not done the preparatory work for lectures are at a very different level from those who have, and that makes it much more difficult to address the needs of all students appropriately. Since both types of students attend the same lectures, variations in the speed at which the instructors review content can hinder either the faster or the slower students. Knutas et al. \cite{knutas2016flipped} draw similar conclusions from their own flipped courses, but they also suggest that these courses offer collaborative exercises during lectures to motivate students to engage with their peers and bring one another to an even level of understanding. One of the main challenges of offering a flipped course is that the gap in understanding between different students must actively be addressed for the entirety of the course, and collaboration among students has been found to be an effective way to bridge this gap.

\section{Combining HyFlex Courses and Flipped Classrooms}

However, when a flipped course also implements the HyFlex model, in which students have limited interaction with their peers, what challenges may arise? There have been studies that examine the intersection between flipped classrooms and HyFlex courses. Griesemer \cite{griesemer2021delivering} conducted a study on their own HyFlex engineering course at a private university during the first year of the pandemic. Though the fact that the course switched from a traditional model to a flipped model halfway through could have provided some unique insight on how students compare the two in the context of a hybrid course, only 19 responses were gathered from students across both sections of the course, drastically reducing the scope of the conclusions drawn in this study. Washuta et al. \cite{washuta2021doing} conducted a similar study that asked students at a military college to rate their experiences with a course that borrows aspects from flipped classroom and hybrid models; however, despite their claim that the course was a HyFlex course, students were nevertheless required to alternate between in-person and remote modalities of attendance, which negates the applicability of their findings for courses that are actually flexible with how students attend. More recently, Nasongkhla and Sujiva \cite{nasongkhla2022hyflex} proposed a framework for flipped HyFlex courses that uses folded origami paper as an analogy for the overlap between different aspects of a course, and they applied this framework to a course of 26 students. Not only was their study quite limited in scope, they only attempted to measure students' creative problem-solving abilities. Metrics that could be used to gauge student engagement levels in the course, such as attendance, were not considered.

Existing studies have neglected to measure the actual attendance counts for the different modalities that are used by students in a flipped HyFlex course, and a study that investigates the decisions made by students in a flipped HyFlex course has yet to take place at a large course at a public research-intensive university. We believe that our study will provide further insight into the day-to-day decision-making process that students undergo when taking a flipped HyFlex course and the particularities that have a positive and negative impact on their perceptions of this model.