\chapter{Introduction}

The unprecedented shift towards online instruction brought about by the COVID-19 pandemic shone a light on the rigidity of the systems that were in place in higher education: remote learning, a modality of learning that had remained outside of the mainstream courses at many higher education institutions, quickly became a necessity. Universities the world over were confronted with the unique challenge of transitioning to an online setting with tools that were not intended to be used for such a format, and they also saw a rise in failure rates, drop rates, and social isolation among their students \cite{lewis2021exploring}. Instructors in fields that had traditionally relied on face-to-face interaction and hands-on exercises were now forced into the difficult position of needing to provide a comparable level of experience to the students of their discipline \cite{gaur2020challenges, serhan2020transitioning}. This period of unexpected adjustment led to widespread exploration of alternative course models.

The college students of today are being exposed to a wide assortment of instructional styles: some courses have returned to traditional \textbf{in-person} instruction, while others have continued to offer \textbf{remote} methods of attendance through online conferencing tools like Zoom and provide video recordings of lectures as an \textbf{asynchronous} option for students. However, while several studies and meta-analysis literature reviews have been conducted on the effects that in-person, remote, and asynchronous modalities can have on students' engagement and educational success \cite{bali2018students, el2007students, jackson2008student, liu2016effectiveness, sharifrazi2019students}, there is a dearth of research on how students interact with courses that combine two models that have seen a rapid increase in popularity: the \textbf{hybrid-flexible (HyFlex)} model, and the \textbf{flipped classroom} model.

The HyFlex model, which was developed by San Francisco State University professor Brian Beatty in 2006 \cite{beatty2014hybrid}, allows students to interact with their lectures in one of three modalities: synchronous in-person, synchronous remote, and asynchronous. The intent of a HyFlex course is to allow students to choose for themselves the educational experience that works best for their individual circumstances, which is more relevant than ever in the aftermath of the COVID-19 pandemic. The flipped classroom model, which can be partially attributed to a paper published by Lage et al. in the year 2000 \cite{lage2000inverting}, provides students with learning resources that can be accessed at any time and are designed to be utilized at one's own pace, and it reorients lectures to serve more as a supplementary resource that reinforces the concepts that students learned ahead of time. Similarly to the HyFlex model, a flipped course is meant to give students more agency in their interactions with the course, and it allows instructors to delve more deeply into course concepts during the lectures than they could if they also needed to introduce those concepts. However, one disadvantage that both models share is that students may not see as much value in attending lectures when they are provided with asynchronous modalities of attendance (in the case of the former) \cite{bali2018students} or resources that can stand on their own as a way to learn the course concepts (in the case of the latter) \cite{campbell2014evaluating}; furthermore, it has yet to be determined conclusively whether combining the two models can exacerbate this impact on student attendance.

Taking all of this into account, our study will analyze the views and decisions formed by students in a flipped HyFlex CS course and explore the following research questions:

% \textbf{RQ1:} Is there a difference in the academic success and engagement level of students who attend lectures through different modalities?

\textbf{RQ1:} What factors impact the modality through which students plan to attend lectures in a flipped HyFlex course?

\textbf{RQ2:} What factors impact the modality through which students attend lectures throughout their enrollment in a flipped HyFlex course?