\chapter{Discussion}

\section{Interpretation of Results}

Through the pre-course survey, we were able to gather demographic data on the students of the course and learn their justifications for enrolling in each of the sections of the course, as well as their overall modality preference. In turn, the mid-course and end-of-course surveys provided us with a way to classify students by their modality of lecture attendance, as well as the reasons that either helped or hindered students when they decided whether to attend lectures in person. By analyzing the responses we gathered through these surveys, we came to the following conclusions for our research questions.

\subsection{Attendance plans are impacted by class standing, less so by gender.}

By stratifying the pre-course survey data, we find that a student's class standing had an impact on their feelings towards in-person and remote modalities: as Figure \ref{fig:modality_preference} shows, students of higher class standings were proportionately more likely to prefer a remote offering over an in-person offering, while the opposite was true of lower class standings. Figure \ref{fig:inperson_enrollment_reasons} shows that second-year undergraduate students who were enrolled in the in-person section were more likely to indicate that they would be on campus when the lectures took place,  and Figure \ref{fig:remote_enrollment_reasons} shows that undergraduate students in their fifth year or beyond were by far the most likely remote enrollees to indicate that they would not be able to get to campus. This may be due, in part, to the fact that on-campus housing is only guaranteed for undergraduate students until the end of their second year, meaning that undergraduate students beyond that point must seek out off-campus options that require them to commute to campus. Further supporting this hypothesis is the fact that graduate students, who are also guaranteed on-campus housing, were not likely to report that they would be unable to get to campus.

Meanwhile, gender appeared to have a less pronounced impact on students' attendance plans at the beginning of the course. Figure \ref{fig:remote_enrollment_reasons} shows that the female students enrolled in the remote section were proportionally less likely to prefer remote modalities than male students. However, only a slightly higher percentage of female students were enrolled in the in-person section. While the reason for this discrepancy is not clearly defined and warrants further study, one interviewee who identified as female offered a possible explanation: they stated that their past experiences with computer science courses had been that male students were the most confident and outspoken, while female students were more hesitant to answer questions during lectures for fear of being wrong. Another female student shared that they had experienced certain uncomfortable interactions with male classmates in the past when attending lectures in-person, and they had thus been more inclined to engage with the course remotely from the start. It appears that the degree to which the field of computer science is male-dominated may have discouraged some female students from considering the in-person modality of the course in spite of their lack of enjoyment for remote learning.

However, unlike class standing and gender, race did not appear to have an impact on students' plans for attendance at the beginning of the course. This may have been due to the fact that the course was heavily skewed towards Asian and White individuals, with members of minority groups not being nearly as represented in our study. This is another potential factor that would benefit from further study, and other researchers would do well to observe a course that has a diverse population of individuals should they conduct a deeper investigation on the impact of race on modality preference.

\subsection{Convenience is key.}

When it came time for students to decide whether to attend lectures in person, they were deterred most often by the early time at which lectures took place: Figure \ref{fig:remote_attendance_reasons} demonstrates how approximately half of the remote, asynchronous, and non-attendees of the course considered the earliness of the lectures to be a factor in their decision to not attend in-person lectures. This, along with the need to commute and the inability to find parking on campus, were all common obstacles mentioned by survey respondents. Often, just one of these factors had been enough to deter a student from heading to the on-campus location from which lectures were held, particularly since remote and asynchronous modalities were accessible from any other location. Interviewees and survey respondents who recounted that they had experienced poor sleep schedules during their time with the course also mentioned that they were also not as motivated to engage with the remote modality for the lectures, choosing to watch the lecture recordings at a later point in the day rather than deprive themselves of more sleep. From these observations, we can conclude that convenience is viewed by many students to be the strongest factor that determines their modality of attendance, even more so than personal preference. Even if students believe that in-person lectures provide a higher-quality educational experience, a course that provides no extrinsic incentive for in-person attendance will likely see few in-person attendees if even one inconveniencing factor is at play.

This desire for convenience extended beyond just the in-person lectures, however: as one respondent to the end-of-course survey put it, having the same content be available across the topic videos, the online textbook, and the lectures made it so that students were less likely to attend lectures remotely as well. Given that the number of remote attendees decreased at a faster rate than the number of in-person attendees, this would imply that students who started to attend remotely eventually found asynchronous recordings to be the most convenient modality of the three, while in-person attendees were more likely to prioritize engagement over convenience. The convenience of being able to access well-designed course material at one's own pace further compounded some students' desires to engage with the course without attending the lectures. From a logistical perspective, having multiple methods through which students can learn about course concepts increases the likelihood that their knowledge of the material will be reinforced; however, it is easy to see how a student can perceive these methods to be redundant and prioritize the most convenient option over all others, even if that method does not allow them to follow up on their misunderstandings and ask clarifying questions as a lecture would. Thus, we hypothesize that while students in HyFlex courses tend to drift towards remote and asynchronous modalities, a flipped HyFlex course is more likely to have students forego synchronous attendance entirely.

\section{Call to Action}

Through the surveys and the follow-up interviews, we identified five main factors that affect the modality through which students attend lectures: personal preference, early expectations, conflicting circumstances, a flexible format, and an insightful instructor. Of the five, the first three were factors that students had already developed by the time they had started the course, meaning that they could not reasonably be changed by those who are managing the course. However, the last two are entirely within the control of the course administrators, particularly the final factor regarding instructor insight. As a result, this section will focus on providing suggestions to instructors and their institutions that offer future HyFlex courses on the ways in which they might be able to influence the modalities through which students attend their lectures. We will not concern ourselves with factors that are out of the control of instructors and instead focus on the aspects that would most benefit from their input.

\subsection{Clearly communicate course logistics.}

As indicated by the high percentage of enrollees in the remote section of the course who did so because the in-person section was full, the fact that both sections of enrollment were functionally identical and met the same course requirements was not well advertised to students prior to the course. Even by the end of the course, four survey respondents expressed that they had not attended lectures in person due to being enrolled in the remote section, showing a lack of communication on behalf of the instructors of the course. Had the department's administration instead offered the course under a single section, the freedom to choose any attendance modality may have come across more clearly to these students. It should be noted that the offering of the course that we observed in our study was the debut of a new course code, which may become less confusing to students of future offerings. Nevertheless, in a course that has multiple modalities through which instructors must communicate with their students, it is imperative that important details about the course are established even before the first lecture is held.

\subsection{Make synchronous lectures more accessible.}

When asked about the reasons why they had not attended lectures in person, two of the remote and asynchronous attendees expressed that they preferred in-person over remote when directly comparing the two modalities, and one even stated that they knew that they would learn better in an in-person environment. However, these respondents faced obstacles that made it difficult for them to choose in-person over remote on most occasions. Factors such as the lengthy commute and limited access to parking exacerbated the rate of decline for in-person attendance to the point where only 16.8\% of students attended lectures in person during the second half of the course. However, the most common of these obstacles was the fact that the lectures were scheduled at 9:00 AM, which these respondents found to be too early, and this even discouraged remote attendees from attending synchronously. As previously mentioned in the Results section, lecture attendance almost always decreases over time in the courses offered by the department, and the fact that students were able to still attend asynchronously at any time and place was cited by many to be the largest benefit of a HyFlex course over previous courses that they had taken. However, the downside of flexibility is that it may discourage students from taking the action that will most benefit their own learning in favor of the one that is most convenient for them.

As a result, instructors would do well to make synchronous modalities more accessible in the ways that are available to them. One possible way to circumvent an early lecture time, for instance, would be to hold other events later in the day that allow students to engage with the course material outside of lectures, such as office hours or tutoring hours. If these events are held in more convenient locations for students, that could also benefit those who struggle to commute and find parking, since those students would have more opportunities to commute to locations that demand a shorter commute or have more available parking spaces. Importantly, the nature of a HyFlex course makes it so that dedicating more attention towards prospective in-person attendees should not have a negative impact on those who would attend remotely or asynchronously regardless of accessibility. The increase in accessibility for prospective in-person attendees should not come at the expense of those using other modalities, nor should in-person attendees receive benefits that are inaccessible to other types of attendees. Meanwhile, institutions should reconsider scheduling HyFlex courses at times that are inconvenient for students, because this is likely to result in most students treating the course as if it were entirely online. The choice, then, is whether it would be more appropriate to offer those courses as online-only courses or to hold them at times that will allow for a much higher number of students to attend synchronously.

\subsection{Practice teaching across multiple modalities.}

Though the interview participants experienced the course through different modalities, one observation that was made by both in-person and remote attendees was that the instructor for the course was experienced in teaching through multiple modalities at once. Their knowledge of Zoom as a teaching tool enabled them to effectively lead a discussion simultaneously across modalities, even as they received questions in person and read comments sent through the online chat. This resulted in an environment in which both in-person and remote attendees felt heard, and this was cited by many survey respondents as something that motivated them to attend lectures synchronously. It is clear that effective management of the modalities results in increased engagement from students, so instructors who are less experienced with the balancing act of HyFlex instruction should take the time to practice doing so. One specific skill that should be focused on is the ability to verbally repeat questions asked by in-person attendees for the benefit of remote and asynchronous attendees and verbally repeat chat messages sent by remote attendees for the benefit of in-person and asynchronous attendees. Instructors should seek out resources that can help them develop this skill; for instance, the Teaching + Learning Commons at the University of California San Diego offers training sessions for instructors on how to use new technologies for their courses. Institutions, in turn, should consider investing in the development of such resources for their instructors, especially if they have yet to do so. Instructors can also have the course's TAs and tutors assist them by facilitating the discussions that take place online, such as by responding to questions that were sent through the chat or guiding students verbally during break-out discussion sections.

\section{Limitations and Future Work}

\subsection{Underrepresented Groups}

There were a few aspects of our study that could use some refinement in future iterations. Most glaringly, as mentioned previously in our discussions of race and gender, there were not as many underrepresented groups in our study as we would have liked due to the demographics of the course being heavily skewed towards Asian, White, and male students. Though we were able to gain further insight on the experiences of female students from the interviews and optional short answer responses, the homogeneity of the student body increases the likelihood that certain observations made in our study will not apply as directly to students of underrepresented groups. Future studies that decide to investigate that direction further should take care to observe a course with a more even distribution of race and gender across its students.

\subsection{Courses with Different Logistics}

In addition, this study was centered around a new offering of a specific course taught by a single instructor at a single institution, which could limit the applicability of our findings to some degree. For example, another course could be taught by multiple instructors and thus provide multiple different opportunities for students to attend lectures live, which may increase the number of synchronous attendees. In addition, the course that we observed is one of the first upper-division courses that are taken by students of the department; it is entirely possible that students in an introductory course or one of the later upper-division courses would require more extrinsic motivators or additional assistance in order to keep being motivated to attend lectures. There are also factors related to both the subject and the students that should be considered: would a theory course with fewer coding-intensive assignments see higher or lower attendance counts than the ones that we observed? If so, which modalities would differ most in this regard? On the other hand, perhaps the recency of the COVID-19 pandemic, which forced many students to enroll in online courses, resulted in students' desires to attend in-person courses to be higher than they would be otherwise. These additional considerations only scratch the surface at the number of possible directions that future studies on HyFlex courses could take.

\subsection{Data Collection Methods}

The methods through which in-person attendance was measured were somewhat imprecise. Pictures and self-reported attendance counts could only capture a proportion of the full group of in-person attendees in the course: the former could not account for individuals who joined the in-person lectures after they began (such as the individual who attended remotely while they conducted their commute to the lecture hall), while the latter was subject to sampling bias and was more representative of those who attended lectures due to the in-class announcements that were made upon the release of the surveys. Because these metrics were less precise than our remote attendance counts, the classifications we designed for attendance types were not as consistent as they could have been, and the overall strength of our data analysis suffered as a result. In the interest of providing a fair experience for students across all modalities, we did not use any methods of tracking the in-person attendance of individual students that impacted grades, such as participation points; however, future studies may be able to gather in-person attendance data through a more efficient method, which would in turn lead to more accurate classifications for attendance types and a subset that is a more accurate representation of the entire course.

\subsection{Directions for Further Research}

The work presented in this paper invites institutions to more intelligently design their flipped HyFlex courses, especially when the needs of students could differ depending on the specific flipped HyFlex course that they are enrolled in. However, there is still much to learn about the intersection between flipped courses and HyFlex courses, and we encourage others to conduct their own studies on courses that combine the two models. Apart from the aforementioned limitations to our study, there are a number of aspects that could be investigated further by researchers in future studies.

One potential direction for further research would be to implement the suggested changes presented above in another flipped HyFlex course and measure whether they have an impact on the utilization level of each attendance modality, particularly in-person attendance. Other researchers may find additional factors impacting attendance that we could not, either due to the limited scope of our study or the setting in which our study took place.

It should also be noted that our study took place only a few years after the COVID-19 pandemic altered the educational paths of many students in institutions around the world. There is reason to believe that the recency of such an impactful event may have caused our data to vary noticably from what future studies may obtain from their inverted HyFlex courses, and that potential discrepancy further supports our call for more research to take place on this topic.

In addition, the only components of our course that were offered in multiple modalities were the lectures and the discussion sections. Other courses may choose to offer tutoring hours or office hours across in-person and remote modalities as well, and the effect that such an addition has on students is a worthy topic for investigation.

We also invite future researchers to make their own observations on flipped HyFlex courses and further investigate the nature of both flipped classrooms and HyFlex. It is our belief that a deeper understanding of course designs like flipped classroom and HyFlex could eventually lead to the development of an educational model that, much like HyFlex itself, brings together various aspects of other models to form a whole that is greater than the sum of its parts.